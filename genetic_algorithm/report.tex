\documentclass{article}
\usepackage{graphicx} % Required for inserting images
\usepackage[colorlinks=true,urlcolor=blue]{hyperref}
\usepackage{algorithm} % Required for the algorithm environment
\usepackage{algorithmic} % Required for the algorithmic environment
\usepackage{amsmath}
\usepackage{algpseudocode}
\usepackage[utf8]{inputenc}
\usepackage[portuguese]{babel}
\usepackage{minted}

\title{GA}
\author{Arthur Valls}
\date{Abril 2024}

\begin{document}

\maketitle

\section{Google Colab}

O código que implementa o GA pode ser encontrado no seguinte  \href{https://colab.research.google.com/drive/1zW4y2xiscGNOEI52n1qlYTZhmqqjBwG9?authuser=1}{link}.

\section{Parâmetros e Operadores}

\section{Gráficos das Gerações}

\begin{figure}[htbp]
  \centering
  \includegraphics[width=0.6\textwidth]{gen0.png}
  \caption{Geração 1}
\end{figure}

\begin{figure}[htbp]
  \centering
  \includegraphics[width=0.6\textwidth]{gen25.png}
  \caption{Geração 25}
\end{figure}

\begin{figure}[htbp]
  \centering
  \includegraphics[width=0.6\textwidth]{gen50.png}
  \caption{Geração 50}
\end{figure}

\begin{figure}[htbp]
  \centering
  \includegraphics[width=0.6\textwidth]{gen100.png}
  \caption{Geração 100}
\end{figure}

\begin{figure}[htbp]
  \centering
  \includegraphics[width=0.8\textwidth]{best2d.png}
  \caption{Melhor a cada geração (2D)}
\end{figure}
\newpage
\section{Pymoo}

A biblioteca pymoo oferece algoritmos de otimização de última geração para problemas de otimização de objetivo único e multiobjetivo, além de muitos outros recursos relacionados à otimização multiobjetivo, como visualização e tomada de decisão \cite{pymoo}. Ela nos fornece o seguinte código de exemplo para executarmos sua implementação do algoritmo genético:

\begin{minted}{python}
from pymoo.algorithms.soo.nonconvex.ga import GA
from pymoo.problems import get_problem
from pymoo.optimize import minimize

problem = get_problem("rosenbrock", n_var=2)

algorithm = GA(pop_size=100)

res = minimize(problem, algorithm,
               ('n_gen', 100),
               verbose=False)

print("Best solution found: \nX = %s\nF = %s" % (res.X, res.F))

\end{minted}

O código acima foi adaptado para utilizar uma função de retorno de chamada, a fim de recuperar os melhores indivíduos de cada geração para a realização dos testes estatísticos que veremos a seguir.

\subsection{Operadores}

O algoritmo genético da \textbf{Pymoo} foi executado com os seguintes operadores, que são os padrões da função:

\begin{itemize}
\item \textbf{Mutação:}
\textbf{Polynomial Mutation (PM)}\cite{pymoo-mutation}

A mutação polinomial tenta simular a distribuição da descendência das mutações de flip de bits codificados em binário com base em variáveis de decisão de valor real. Essa mutação favorece a prole mais próxima do progenitor\cite{r-pm-mutation}. 
    
\item \textbf{Seleção:} \textbf{Tournament Selection}\cite{pymoo-selection}

Similar ao operador que utilizei, dois indivíduos são selecionados e o que tem a melhor \textbf{fitness} é escolhido para gerar a prole.

\item \textbf{Cruzamento:} \textbf{Simulated Binary Crossover (SBX)}\cite{pymoo-crossover}

Um operador de recombinação de parâmetros reais comumente usado na literatura de algoritmo evolutivo (EA). O operador envolve um parâmetro que dita a dispersão das soluções da prole em relação às soluções dos progenitores \cite{deb2007self}. 

\end{itemize}


\subsection{Parâmetros}

A documentação do site da Pymoo (\url{pymoo.org}) oferece uma explicação detalhada sobre a aplicação de cada operador, porém não especifica muito sobre os parâmetros selecionados para cada um deles. No entanto, ao entrar no GitHub da organização (\url{https://github.com/anyoptimization/pymoo/tree/main}), podemos pesquisar por cada módulo individualmente e encontrar alguns dos parâmetros utilizados:


\begin{itemize}
    \item \textbf{Cruzamento (SBX)}

   
 \texttt{prob\_var}: 0.5 \\
        Probabilidade de aplicar o SBX em cada variável.
  
  \texttt{prob\_exch}: 1.0 \\
        Probabilidade de troca de informações entre indivíduos.
  
  \texttt{prob\_bin}: 0.5 \\
        Probabilidade de realizar crossover binário.
  
  \texttt{n\_offsprings}: 2 \\
        Número de soluções descendentes geradas por crossover.

  \item \textbf{Mutação (PM)}
  
     \texttt{eta}: 20\\
     Representa o parâmetro de distribuição polinomial para cada solução na população. Este parâmetro controla a extensão da mutação.
   
     \texttt{prob}: 0.9\\
     Representa a probabilidade de mutação para cada solução na população.
   
     \texttt{at\_least\_once}: False\\
     Parâmetro booleano opcional que determina se pelo menos uma variável deve ser mutada.
  
\end{itemize}

Os demais parâmetros, como \textbf{limite superior e inferior de busca}, \textbf{tamanho da população}, \textbf{número de gerações} e \textbf{dimensões do problema}, foram ajustados para corresponder à minha implementação, visando realizar uma comparação justa.

\section{Comparação entre os métodos}

Com o objetivo de avaliar a eficiência de ambas as implementações, foram realizadas 30 execuções de cada método com 100 indivíduos e 100 gerações em três cenários dimensionais: 2, 5 e 10. 


\subsection{2 Dimensões}

\begin{figure}[htbp]
  \centering
  \includegraphics[width=0.51\textwidth]{conv2d.png}
  \caption{Média dos melhores indivíduos por geração após 30 execuções dos algoritmos.}
\end{figure}

\begin{table}[htbp]
    \centering
    \begin{tabular}{|l|c|c|}
    \hline
    \textbf{} & \textbf{Arthur} & \textbf{Pymoo} \\
    \hline
    \textbf{Média} & $6.652122831398276 \times 10^{-5}$ & $9.18109627834858\times 10^{-5}$ \\
    \hline
    \textbf{Desvio Padrão} & 0.00016299117605273977 & 0.00010447352060452732 \\
    \hline
    \textbf{Mediana} & $1.7275937063029017\times 10^{-8}$ & $5.6292959826598995\times 10^{-8}$ \\
    \hline
    \end{tabular}
    \caption{Estatísticas dos melhores indivíduos da última geração de cada uma das 30 execuções.}
\end{table}

\newpage
\bibliographystyle{plain}
\bibliography{reference} % nome do arquivo .bib (sem a extensão)

\end{document}
